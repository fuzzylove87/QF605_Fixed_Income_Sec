\documentclass{article}
\usepackage{amsmath}
\usepackage{amssymb}
\usepackage{parallel}
\usepackage[most]{tcolorbox}
\usepackage{amsthm}
\usepackage{caption}
\usepackage{subcaption}
\usepackage[a4paper,
top=1.5cm,
bottom=1.5cm,
left=1.8cm,
right=1.8cm,
heightrounded]
{geometry}
\renewcommand{\arraystretch}{2.5}

\begin{document}
	
	
\begin{titlepage}
	\centering
	\vspace*{\fill}
	
	\vspace*{0.5cm}
	
	\huge\bfseries
	QF605 \\
	Fixed Income Securities \\ 
	Group Project \\
	
	\vspace*{0.5cm}
	
	\large Team members: \\ 
	
	\vspace*{0.3cm}
	
	\large Jin Weiguo, Kim Chan-Jung, Johnny Quek, \\
	\large Wang Boyu, Woon Tian Yong \\
	
	\vspace*{\fill}
	
\end{titlepage}
\newpage	
% PART 1: Bootstrappng Swap curves %%%%%%%%%%%%%%%%%%%%%%%%%%%%%%%%%%%%%%%%%%%%%%%%
\par \noindent \textbf{Part I: Bootstrapping Swap Curves}\\
\noindent \textbf{OIS Discount Factors}\\

\noindent With the provided OIS rates data, we proceeded to use the following methodology to bootstrap the OIS discount factor curve. 
\begin{flalign*}
PV_{fix} &= PV_{float} \\
D(0,1y) * OIS_{1y} &= D(0,1y) * [ (1 + \frac{f_0}{360})^{360} - 1 ] \\
[D(0,1y) + D(0,2y)] * OIS_{2y} &= D(0,1y) * [(1 + \frac{f_0}{360})^{360} - 1] + D(0,2y) * [(1 + \frac{f_1}{360})^{360} - 1] \\
&\vdots \\
[D(0,1y) + \dots + D(0,20y)] * OIS_{20y} &= D(0,1y) * [(1 + \frac{f_0}{360})^{360} - 1] + \dots + D(0,20y) * [(1 + \frac{f_{19}}{360})^{360} - 1] 
\end{flalign*} 

\noindent Due to only a handful of OIS swaps observable in the market, we can only then use the OIS swaps of varying tenor [6m,1y,2y,3y,5y,7y,10y,20y] to bootstrap the whole OIS discount curve while linearly interpolate for the rest of the "gap" discount factors. In order to solve for all discount factors, we will have to adopt the following: 
\begin{flalign*}
PV_{fix} &= PV_{float} \\
[D(0,1y) + \dots + D(0,7y)] * OIS_{7y} &= D(0,1y) * [(1 + \frac{f_0}{360})^{360} - 1] + \dots + D(0,7) * [(1 + \frac{f_{6}}{360})^{360} - 1] 
\end{flalign*} 

\noindent* Assuming all prior discount factors have been bootstrapped, we can then subtitute the following into above equation to help isolate D(0,7y), and then derive D(0,7y): 

\begin{flalign*}
f_6 &= 360 * [D(0,7y)^{-\frac{1}{360*7}} - 1]   \\
D(0,6y) &= \frac{[D(0,7y) - D(0,5y)]}{2}*1 + D(0,5y)\\
f_5 &= 360 * [[\frac{[D(0,7y) - D(0,5y)]}{2}*1 + D(0,5y)]^{-\frac{1}{360*6}} - 1]
\end{flalign*}

\noindent Proceeding to do the similar for all OIS swaps, we derive the following OIS discount factors results and graph.

\begin{figure}[h]
	\centering
	\begin{subfigure}{.5\textwidth}
		\centering
		\includegraphics[width=.75\linewidth]{./images/OIS_df.jpg}
		\caption{OIS Discount Curve}
		\label{fig:sub1}
	\end{subfigure}%
	\begin{subfigure}{.5\textwidth}
		\centering
		\includegraphics[width=.5\linewidth]{./images/OIStable.jpg}
		\caption{OIS Discount Factors}
		\label{fig:sub2}
	\end{subfigure}
	\caption{OIS Results}
	\label{fig:test}
\end{figure}

\noindent \textbf{LIBOR Discount Factors}\\

\noindent Similarly for LIBOR discount factors, we will adopt the same approach using the OIS discount factors to derive with the forward libor rates and LIBOR discount factors.

\begin{flalign*}
PV_{fix} &= PV_{float} \\
0.5 * [D_{OIS}(0,0.5y) + D_{OIS}(0,1y) ] * IRS_{1y} &= 0.5 * [D_{OIS}(0,6m)*L(0,6m)+D_{OIS}(0,1y)*L(6m,1y)] \\
&\vdots \\
0.5 *[D_{OIS}(0,0.5y) + \dots + D_{OIS}(0,20y)] * IRS_{20y} &= 0.5 * [D_{OIS}(0,0.5y) * L(0,6m) + \dots + D_{OIS}(0,20y) * L(19.5y,20y)]
\end{flalign*} 

\noindent Likewise, we will also substitute the following equations to solve for one unknown (ie. LIBOR discount factor) for each of the above equation starting from 0.5y\dots20y.In this example, we will use 7 years IRS to be consistent with our OIS approach

\begin{flalign*}
D(0,5.5y) &= [\frac{[D(0,7y) - D(0,5y)]}{4} * 1 + D(0,5y)] * \frac{1}{\frac{1}{2}} \\
D(0,6y) &= [\frac{[D(0,7y) - D(0,5y)]}{4} * 2 + D(0,5y)] * \frac{1}{\frac{1}{2}} \\
D(0,6.5y) &= [\frac{[D(0,7y) - D(0,5y)]}{4} * 3 + D(0,5y)] * \frac{1}{\frac{1}{2}} \\
L(5y,5.5y) &= \frac{D(0,5y) - D(0,5.5y)}{D(0,5.5y)} * \frac{1}{\frac{1}{2}}    \\
L(5.5y,6y) &= \frac{D(0,5.5y) - D(0,6y)}{D(0,6y)}  * \frac{1}{\frac{1}{2}}    \\
L(6y,6.5y) &= \frac{D(0,6y) - D(0,6.5y)}{D(0,6.5y)}  * \frac{1}{\frac{1}{2}}    \\
L(6.5y,7y) &= \frac{D(0,6.5y) - D(0,7y)}{D(0,7y)}   * \frac{1}{\frac{1}{2}}   \\
\end{flalign*}


\noindent Proceeding to execute the same approach for all IRS , we derive the following LIBOR discount factors results and graph.

\begin{figure}[h]
	\centering
	\begin{subfigure}{.5\textwidth}
		\centering
		\includegraphics[width=.75\linewidth]{./images/LIBOR_df.jpg}
		\caption{LIBOR Discount Curve}
		\label{fig:sub1}
	\end{subfigure}%
	\begin{subfigure}{.5\textwidth}
		\centering
		\includegraphics[width=.5\linewidth]{./images/LIBORtable.jpg}
		\caption{LIBOR Discount Factors}
		\label{fig:sub2}
	\end{subfigure}
	\caption{LIBOR Results}
	\label{fig:test}
\end{figure}


\noindent \textbf{Forward Swap Rates}\\

\noindent With all the necessary OIS discount factors and Forward LIBOR rates, we can go on to derive the Forward Swap rates:

\begin{figure}[ht]
	\centering
	\includegraphics[width= \linewidth]{./images/FwdSwaps.jpg}
	\caption{Forward Swap rates}
\end{figure}

\newpage
	
\par \noindent \textbf{Part II: Swaption Calibration}\\
\noindent \textbf{Model Calibration}

\begin{figure}[h]
	\centering
	\begin{subfigure}{.5\textwidth}
		\centering
		\includegraphics[width=1\linewidth]{./images/DD.jpg}
		\caption{Displaced-Diffsion Model}
		\label{fig:sub1}
	\end{subfigure}%
	\begin{subfigure}{.5\textwidth}
		\centering
		\includegraphics[width=0.7\linewidth]{./images/SABR.jpg}
		\caption{SABR Model}
		\label{fig:sub2}
	\end{subfigure}
	\caption{Parameter Calibration}
	\label{fig:test}
\end{figure}

\noindent \textbf{Pricing swaptions using the calibrated model}\\
\begin{figure}[h]
	\centering
	\begin{subfigure}{.5\textwidth}
		\centering
		\includegraphics[width=0.9\linewidth]{./images/Payer.png}
		\caption{payer 2y x 10y}
		\label{fig:sub1}
	\end{subfigure}%
	\begin{subfigure}{.5\textwidth}
		\centering
		\includegraphics[width=0.91\linewidth]{./images/Receiver.png}
		\caption{receiver 8y x 10y}
		\label{fig:sub2}
	\end{subfigure}
	\caption{Swaption price}
	\label{fig:test}
\end{figure}

\noindent We are calibrating sigma and beta in the meantime since both Black Scholes and Bachelier model assumes constant $\sigma$. To this end, we applied least square method to find optimal sigma and beta such that deviation of price given by DD-model and actual price is minimized. \\

\noindent In order to price payer swap 2y*10y, we interpolate sigma and beta between 1y*10y and 5y*10y DD-model after which calibrated model is used for pricing. When pricing through SABR, we interpolate alpha, rho and nu to get the right model for 2y expiry. We do the same for 8y receiver swap expect for interpolation is conducted between 5y expiry and 10 expiry now.

\newpage
\noindent \textbf{Fitting curve}\\
\begin{figure}[h]
	\flushleft
	\includegraphics[width=1\linewidth]{./images/1Y.png}
	\caption{1y expiry swaption: Tenor from 1y to 10y}
	\label{fig:sub1}
\end{figure}
\begin{figure}[h]
	\flushleft
	\includegraphics[width=1\linewidth]{./images/5Y.png}
	\caption{5y expiry swaption: Tenor from 1y to 10y}
	\label{fig:sub1}
\end{figure}
\begin{figure}[h]
	\flushleft
	\includegraphics[width=1\linewidth]{./images/10Y.png}
	\caption{10y expiry swaption: Tenor from 1y to 10y}
	\label{fig:sub1}
\end{figure}


\noindent According to the fitting curve, the SABR model can fitting the volatility "smile" closely for every expiry and tenor type of swaption, but for Displaced-Diffusion Model, the least sqaue method calibrated constant $\sigma$ and $\beta$ make the implied volatiliy only fit the market volatility roughly and couldn't keep trace to the existing smlie. \\ 


\noindent Given the same expiry, the longer the tenor is, the more implied volatility close to lognormal distributed. And assume the same tenor, $\beta$ also goes up with expiry of the contract increase.

\newpage
	
\par \noindent \textbf{Part III - Convexity Correction}\\


\noindent \textbf{Present value of CMS product}	\\ \\                     
\noindent To calculate PV of leg receiving CMS10y semi-annually over the next 5 years, we need to find SABR paramters at different expiries in order to price each CMS rate. To this end, cubic spline interpolation is used between $\alpha$ $\nu$, and $\rho$ of $1y$ $\times$ $10y$, $5y\times10y$ and $10y\times10y$ SABR models we have calibrated. Since there was no expiry lower than 1y for us to interpolate, parameters for 0.5y expiry follows those of 1y expiry.\\

\noindent Interpolation profiles are given as follow:

\begin{figure}[h]
	\centering
	\includegraphics[scale=0.5]{Cubic_10y.png}
	\caption{Interpolation of SABR parameters - CMS10Y}
\end{figure}	

\noindent After interpolating all the SABR parameters, static replication is used to price each CMS rate and PV is the sum of the discounted values of all CMS rates, multiplied by the day count fraction. Here goes the mathematical form:
\begin{align*}
PV_{CMS10y}&=D(0,6m)\times 0.5 \times E^T [S_{6m,10y6m}(6m)] \\&+ D(0,1y) \times 0.5 \times E^T [S_{1y,11y}(1y)]\\&+ \dots 
+D(0,5y) \times 0.5 \times E^T [S_{5y,15y}(5y)]\\&= 0.213606
\end{align*}



\noindent Similarly, for CMS2y processed quarterly, $\alpha$ $\nu$, and $\rho$ can be interpolated between $1y$ $\times$ $2y$, $5y\times2y$,$10y\times2y$, whose profiles are demonstrated below:

\begin{figure}[h]
	\centering
	\includegraphics[scale=0.5]{Cubic_2y.png}
	\caption{Interpolation of SABR parameters - CMS2Y}
\end{figure}	

\noindent In addition to SABR parameters interpolation, due to quarterly arrangement, more discrete OIS discount rates and Libor discount rates are interpolated based on DF calculated in Section 1. After getting all the inputs, we can calculate PV of CMS2y as follow:
\begin{align*}
PV_{CMS2y}&=D(0,3m)\times 0.25 \times E^T [S_{3m,2y3m}(3m)] \\&+ D(0,6m) \times 0.25 \times E^T [S_{6m,2y6m}(6m)]\\&+ \dots 
+D(0,10y) \times 0.25 \times E^T [S_{10y,12y}(10y)]\\&= 0.504841
\end{align*}

\newpage

\noindent \textbf{CMS VS Par Swap Rate}\\ \\
Through trial and error, we found out that the CMS rates can become unlikely large numbers or even drop below par swap rate when upper bound for payer swaption integral is set as a large number or infinity. To figure out the optimal upper bound that not only covers most of the cases but generates plausible CMS rates, we calculated pure $f(K)$ in CMS rate formula by setting the payoff function as constant 1. When the upper bound is 0.85, no value exceeded 1, and CMS rates converged on reasonable readings.

\begin{figure}[h]
	\centering
	\includegraphics[scale=0.48]{Coverage.png}
	\caption{Coverage of Integral inside of CMS rate when Upper Bound is 0.85}
\end{figure}

Tables presented below show CMS rates for each maturity and tenor. 

\begin{figure}[h]
	\centering
	\includegraphics[scale=0.40]{CMS_RATE.png}
	\caption{CMS rates}
\end{figure}

\noindent Comparing CMS rates with forward swap rates of corresponding expiry and tenor which are derived from Part 1, we can recognise that the difference between CMS and forward swap rate increases as the expiry lengthens. It means that the longer expiry becomes, the greater the magnitude of convexity correction grows. On the contrary, the influence of tenor on the convexity correction is irregular. This phenomenon is presumed to be a result of volatility smile. 

\begin{figure}[ht]
	\centering
	\includegraphics[scale=0.45]{CMS_FSR.png}
	\caption{Delta Profile for Up-and-In Barrier Option of Given Condition}
\end{figure}

\newpage

\par \noindent \textbf{Part IV - Decompounded Options}\\
\par \noindent \textbf{Question 1}\\

\noindent Starting from the generic contract valuation formula, we were able to obtain the static replication formula for the contract in question by first applying Leibniz's Rule on the IRR Payer and Receiver swaption formulas twice, after which integration by parts was carried out twice on the integrals in the generic contract valuation formula. \\

\noindent First, applying Leibniz's rule on the IRR Payer and Receiver swaption formulas twice yields:\\ \\
\noindent
\begin{minipage}[c]{0.5\textwidth}
	\begin{tcolorbox}[height=3.5cm,boxsep=5pt,arc=0pt,auto outer arc,colback=white,colframe=black]
		\noindent \textbf{Payer IRR Swaption}
		$$V^{pay}(K) = D(0,T) \int_{K}^{\infty} IRR(S) \cdot (S-K) \cdot f(S) dS$$
		$$\frac{\partial^2 V^{pay} (K)}{\partial K^2} = D(0,T) \cdot IRR(K) \cdot f(K)$$
	\end{tcolorbox}
\end{minipage}
\begin{minipage}[c]{0.5\textwidth}
	\begin{tcolorbox}[height=3.5cm,boxsep=5pt,arc=0pt,auto outer arc,colback=white,colframe=black]
		\noindent \textbf{Receiver IRR Swaption}
		$$V^{rec}(K) = D(0,T) \int_{0}^{F} IRR(S) \cdot (K-S) \cdot f(S) dS$$
		$$\frac{\partial^2 V^{rec} (K)}{\partial K^2} = D(0,T) \cdot IRR(K) \cdot f(K)$$
	\end{tcolorbox}
\end{minipage}\\ \\

\noindent We are now able to denote the generic contract valuation formula as such:
\begin{flalign*}
V_0 &= D(0,T)\mathbb{E} \left[ g(S) \right]\\
&=D(0,T) \int_{0}^{\infty} g(K) f(K) dK\\
&= \int_{0}^{F} h(K) \frac{\partial^2 V^{rec} (K)}{\partial K^2} dK + \int_{F}^{\infty} h(K) \frac{\partial^2 V^{pay} (K)}{\partial K^2} dK
\end{flalign*}\\

\noindent Integration by parts twice yields:
$$ V_0=D(0,T)g(F) + \int_{0}^{F} h''(K) V^{rec}(K)dK + \int_{F}^{\infty} h''(K) V^{pay}(K)dK $$

\noindent Where:\\
$$ F=S_{n,N}(0), \quad n=5, \quad N=15, \quad T=5 $$

$$ g(K) = K^{\frac{1}{p}} - 0.04^{\frac{1}{q}} = K^{\frac{1}{4}}-0.2, \quad
g'(K) = \frac{1}{4} K^{-\frac{3}{4}}, \quad
g''(K) = -\frac{3}{16}K^{-\frac{7}{4}} $$

$$ h(K) = \frac{g(K)}{IRR(K)}, \quad
h'(K) = \frac{IRR(K)g'(K)-g(K)IRR'(K)}{IRR(K)^2} $$

$$ h''(K) = \frac{IRR(K)g''(K)-IRR''(K)g(K)-2IRR'(k)g'(K)}{IRR(K)^2} + \frac{2IRR'(K)^2g(K)}{IRR(K)^3} $$

$$ V^{rec}=D(0,T) \cdot IRR(S_{n,N}(0)) \cdot \textnormal{Black76Put}(S_{n,N}(0),K,\sigma_{\textnormal{SABR}},T) $$

$$ V^{pay}=D(0,T) \cdot IRR(S_{n,N}(0)) \cdot \textnormal{Black76Call}(S_{n,N}(0),K,\sigma_{\textnormal{SABR}},T) $$\\

\noindent Using these parameters, we were able to obtain a Present Value (PV) for the $V_0$ of \textbf{0.2334}. $V_0$ can also be seen as a forward contract on the 10-year CMS rate with forward price set at 0.0016.

\newpage

\par \noindent \textbf{Question 2}\\

\noindent The contract ($V_0^+$) can be valued as though it is a CMS caplet when the payoff is $(S_T^{1/4}-0.04^{1/2})^+$.\\ \\
\noindent For $S_T^{1/4}-0.04^{1/2}$ to be positive:
\begin{flalign*}
S_T^{1/4}& >0.2\\
S_T &> 0.0016 = L
\end{flalign*}

\noindent Thus, we can see $V_0^+$ as a CMS caplet struck at $L=0.0016$:

\begin{flalign*}
V_0^+ &= D(0,T) \int_{L}^{\infty} g(K)f(K) dK\\
&=\int_{L}^{\infty} h(K) \frac{\partial^2 V^{pay }(K)}{\partial K^2} dK\\
&= h'(L) V^{pay} (L) + \int_{L}^{\infty} h''(K) V^{pay}(K)dK
\end{flalign*}

\noindent Using this valuation formula and relevant parameters from Question 1, we were able to obtain a Present Value (PV) for $\boldsymbol{V_0^+}$\textbf{ of 0.2430}, which is higher than $\boldsymbol{V_0}$\textbf{'s value of 0.2334}. \\ \\

\noindent This is intuitive as $V_0^+$ omits the negatively valued region of $V_0$ where $S_T < L$ and as such should be valued higher than $V_0$. It also follows that $V_0<V_0^+< \mathbb{E} \left[S_T^{1/4}\right]$ (spot price of the underlying), as this is the model-free no-arbitrage boundary that the three products must satisfy. Running further diagnostics, it can be seen that $V_0^+$ is consistently more highly priced than $V_0$ across the given values of $N$ (swap end date) for every $n$ (swap start date):

\begin{figure}[h]
	\centering
	\subcaptionbox{\label{}}{\includegraphics[width=.48\linewidth]{./P4n1.png}}\hspace{0.05em}
	\subcaptionbox{\label{}}{\includegraphics[width=.48\linewidth]{./P4n5.png}}\hspace{0.05em}
	\subcaptionbox{\label{}}{\includegraphics[width=.48\linewidth]{./P4n10.png}}
\end{figure}


\end{document}